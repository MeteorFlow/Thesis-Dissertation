Database systems perform vital functions for all sorts of organizations
because of the growing importance of using and managing data efficiently. A
database system consists of a software, a database management system (DBMS)
and one or several databases. DBMS is a set of programs that enables users to
store, manage and access data. In other words database is processed by DBMS,
which runs in the main memory and is controlled by the respective operating
system

A database is a logically coherent collection of data with some inherent
meaning and represents some aspects of the real world. A random assortment of
data cannot be referred to as a database. Databases draw a sharp distinction
between data and information. Data are known facts that can be recorded and that
have implicit meaning. Information is data that have been organized and prepared
in a form that is suitable for decision-making. Shortly information is the analysis
and synthesis of data.
The most fundamental terms used in database approach are ìentityî,
ìattributeî and ìrelationshipî. An entity is something that can be identified in the
usersí work environment, something that the users want to track. It may be an
object with a physical or conceptual existence. An attribute is a property of an
entity. A particular entity will have a value for each of its attributes. The attribute
values that describe each entity become a major part of data stored in the database 

Database Management System is a general-purpose software system
designed to manage large bodies of information facilitating the process of
defining, constructing and manipulating databases for various applications.
Specifying data types, structures and constraints for the data to be stored in the
database is called defining a database. Constructing the database is the process of
storing data itself on some storage medium that is controlled by the DBMS.
Querying to retrieve specific data, updating the database to reflect changes and
generating reports from the data are the main concepts of manipulating a database.
The DBMS functions as an interface between the users and the database
ensuring that the data is stored persistently over long periods of time, independent
of the programs that access it \cite{latisen1998}.
DBMS can be divided into three subsystems; the design tools subsystem,
the run time subsystem and the DBMS engine. 

The design tools subsystem has a set of tools to facilitate the design and
creation of the database and its applications. Tools for creating tables, forms,
queries and reports are components of this system. DBMS products also provide
programming languages and interfaces to programming languages. The run time
subsystem processes the application components that are developed using the
design tools. The last component of DBMS is the DBMS engine which receives
requests from the other two components and translates those requests into
commands to the operating system to read and write data on physical media \cite{elmasri1998}.

Database approach has several advantages over traditional file processing
in which each user has to create and define files needed for a specific application.
In these systems' duplication of data is generally inevitable causing wasted storage
space and redundant efforts to maintain common data up-to date. In database
approach data is maintained in a single storage medium and accessed by various
users. The self-describing nature of database systems provides information not
only about database itself but also about the database structure such as the type and
format of the data. A complete definition and description of database structure and
constraints, called meta-data, is stored in the system catalog. Data abstraction is a
consequence of this self-describing nature of database systems allowing programdata independence. 
DBMS access programs do not require changes when the structure of the data files are changed hence 
the description of data is not embedded in the access programs. This property is called program-data
independence. Support of multiple views of data is another important feature of
database systems, which enables different users to view different perspective of
database dependent on their requirements. In a multi-user database environment
users probably have access to the same data at the same time as well as they can
access different portions of database for modification. Concurrency control is
crucial for a DBMS so that the results of the updates are correct. The DBMS
software is to ensure that concurrent transactions operate correctly when several
users are trying to update the same data

Using a DBMS also eliminates unnecessary data redundancy. In database
approach each primary fact is generally recorded in only one place in the database
[6]. Sometimes it is desirable to include some limited redundancy to improve the
performance of queries when it is more efficient to retrieve data from a single file
instead of searching and collecting data from several files, but this data duplication
is controlled by DBMS so as to prohibit inconsistencies among files. By
eliminating data redundancy inconsistencies among data are also reduced \cite{elmasri1998}.
Reducing reduncancy improves the consistency of data while reducing the waste
in storage space. DBMS gives the opportunity of data sharing to the users. Sharing
data often permits new data processing applications to be developed without
having to create new data files. In general, less redundancy and greater sharing
lead to less confusion between organizational units and less time spent resolving
errors and inconsistencies in reports. The database approach also permits security
restrictions. In a DBMS different types of authorizations are accepted in order to
regulate which parts of the database various users can access or update. 