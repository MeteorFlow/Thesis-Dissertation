Hệ thống nền tảng dữ liệu thời tiết (Weather Data Platform - WDP) được phát triển với mục tiêu trở thành một giải pháp toàn diện cho việc khai thác sức mạnh của dữ liệu thời tiết. Hệ thống được thiết kế để đáp ứng những yêu cầu cụ thể của các nhà khí tượng, học giả, nghiên cứu học thuật và các nhà phát triển từ nhiều lĩnh vực khác nhau, bao gồm cả freelancers, doanh nghiệp và tổ chức phi chính phủ (Non-governmental Organizations - NGOs). WDP đóng vai trò như một trung tâm tập trung cho việc tích hợp dữ liệu thời tiết, phân tích, và nhiều tính năng khác.

Với mong muốn phát triển thành Hệ thống nền tảng dữ liệu thời tiết, chúng tôi hướng đến sự hoàn thiện và đa chiều hoá thông tin thời tiết. Không chỉ là một bảng số liệu, mà là một trải nghiệm toàn diện. Trong tương lai, bên cạnh việc tiếp tục xây dựng cơ sở dữ liệu tích hợp theo hướng đã đề xuất, chúng tôi hứa hẹn sẽ tiếp tục nghiên cứu để mở rộng và phát triển cơ sở dữ liệu tích hợp này thành hệ thống nền tảng dữ liệu thời tiết với những hướng phát triển như sau:
\begin{enumerate}
    %\item Sử dụng các phương pháp quy trình phát triển phần mềm (SDLC) như Agile để phát triển và giao sản phẩm một cách hiệu quả.
    %\item Phân biệt rõ giữa các tính năng cốt lõi và mở rộng.
    %\item Quyết định liệu sản phẩm có nên có khả năng cố định hay có thể mở rộng.
    %\item Theo dõi và áp dụng các công nghệ mới trong lĩnh vực thời tiết và dự báo để cải thiện chất lượng dữ liệu và dự báo.
    \item \textbf{Dữ liệu phi tuyến:} Mở rộng từ việc tích hợp dữ liệu cơ bản, chúng tôi sẽ chú trọng vào việc cung cấp dữ liệu phi tuyến, chi tiết và đa nguồn, giúp người dùng khám phá thêm về môi trường xung quanh.
    \item \textbf{Trí tuệ nhân tạo thấu hiểu}: Sử dụng trí tuệ nhân tạo để thấu hiểu ngôn ngữ của thời tiết, từ những biến đổi nhỏ đến những sự kiện lớn, tạo nên một nguồn thông tin thời tiết sâu sắc và thông minh.
    \item \textbf{Giao diện người dùng tương tác}: Không chỉ là việc truy cập thông tin, mà còn là việc tương tác với dự báo thời tiết. Giao diện người dùng sẽ là nơi người dùng thể hiện sự tò mò và tương tác trực tiếp với dữ liệu.
    \item \textbf{Kết Nối Thông Tin Địa Lý}: Tận dụng hệ thống thông tin địa lý để mang đến cái nhìn thực tế hóa, địa bàn hóa cho dự báo thời tiết. Điều này giúp người dùng hiểu rõ hơn về tác động thời tiết đối với môi trường xung quanh họ.
    %\item Xây dựng một nền tảng tích hợp dữ liệu thời tiết, phân tích và các nguồn dữ liệu khác.
    \item \textbf{Tối ưu hoá hiệu suất}: đảm bảo khả năng đáp ứng nhanh chóng và đồng đều trong mọi điều kiện.
    \item \textbf{Bảo mật dữ liệu}: Tăng cường an toàn dữ liệu để đảm bảo tính bảo mật và toàn vẹn của thông tin thời tiết.
    \item \textbf{Hệ thống dự báo nâng cao}: Nghiên cứu và tích hợp trí tuệ nhân tạo để cải thiện khả năng dự báo và đưa ra thông tin dự báo cáo chính xác.
    \item \textbf{Kiểm thử và tối ưu hoá}: Tiến hành kiểm thử hệ thống để đảm bảo tính ổn định và xử lý mọi vấn đề tiềm ẩn. Tối ưu hóa hiệu suất nếu cần.
    \item \textbf{Triển khai và duy trì}: Triển khai hệ thống và duy trì một chu kỳ cập nhật đều đặn để đảm bảo rằng nó luôn cung cấp thông tin thời tiết chính xác và đáng tin cậy.
\end{enumerate}