
\subsection{Programming languages and Libraries}

C\# was selected for this project because of its widespread popularity in
enterprise applications and its capability to operate across various operating
systems following the release of .NET Core. This feature significantly enhances
deployment flexibility, allowing the application to be utilized in diverse
environments. 

Additionally, we have incorporated ASP.NET Core specifically version 8 into our
technology stack.ASP.NET Core is an open-source, high-performance framework for
building modern, cloud-based, internet-connected applications. By leveraging
ASP.NET Core, we benefit from a robust, well-supported framework that
facilitates the development of scalable and secure web applications. It
seamlessly integrates with C\#, enabling us to utilize a consistent programming
environment while also exploiting features such as dependency injection, a vast
ecosystem of middleware, and a strong configuration system that is suited to
modern web applications.

ASP.NET Core 8 brings forward improvements in areas such as minimized startup
times, reduced memory footprint, and enhanced security features, making it an
ideal choice for developing scalable and secure web applications. The choice
also underscores our commitment to developing applications that are both
efficient and future-proof, ensuring that they perform optimally on both Windows
and non-Windows platforms. This alignment with .NET Core’s cross-platform
capabilities ensures that our project remains versatile and adaptable to the
evolving technological landscape.

Optionally, we have integrated the Nuxt framework into our technology stack.
Nuxt is a progressive Vue framework that is used for building more robust and
versatile web applications. It simplifies the development process by handling
various aspects of the web infrastructure, such as server-side rendering, static
site generation, and automatic code splitting. This inclusion enriches our
application's interactivity and user experience, providing a seamless and
dynamic interface for users.

\subsection{Command Query Responsibility Segregation}

The Command Query Responsibility Segregation (CQRS) is an architectural pattern
that distinctively separates the tasks of reading data (queries) and writing
data (commands) within a software application. This separation splits
responsibilities into two main components:

\begin{itemize}
\item \textbf{Command Side}: This component manages operations that modify the
system's state. It handles incoming commands from clients or external systems,
conducts validations, and updates the data store accordingly. This side is
essential for maintaining the integrity and accuracy of data modifications
within the application.
\item \textbf{Query Side}: Dedicated to data retrieval, this component processes
all read requests. It fetches data from the appropriate sources, ensuring that
the information provided is accurate and reflects the current state of the data
store.
\end{itemize}

Key advantages of employing the CQRS pattern are:

\begin{itemize}
\item \textbf{Scalability}: CQRS allows for the independent scaling of the read
and write components based on their respective workloads, which can
significantly enhance system performance.
\item \textbf{Flexibility}: With the separation of concerns, different storage
and optimization strategies can be applied to the reading and writing processes.
This flexibility enables the use of the most appropriate tools for each
function, optimizing efficiency.
\item \textbf{Event-Driven Architecture Compatibility}: The use of CQRS often
complements event-driven architectures, where changes in the system's state are
captured and managed as events. This compatibility ensures that the architecture
is dynamic and responsive to changes in business requirements.
\end{itemize}

