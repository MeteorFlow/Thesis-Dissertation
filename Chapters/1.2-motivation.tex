Information about the weather has been recorded in manuscript form for many centuries. The
early records included notes on extreme and, sometimes, catastrophic events and also on
phenomena such as the freezing and thawing dates of rivers, lakes and seas, which have taken
on a higher profile with recent concerns about climate change.
Specific journals for the collection and retention of climatological information have been used
over the last two or three centuries (WMO 2005). The development of instrumentation to quantify
meteorological phenomena and the dedication of observers to maintaining methodical, reliable
and well-documented records paved the way for the organized management of climate data.
Since the 1940s, standardized forms and procedures gradually became more prevalent and,
once computer systems were being used by NMHSs, these forms greatly assisted the
computerized data entry process and consequently the development of computer data archives.
The latter part of the twentieth century saw the routine exchange of weather data in digital form
and many meteorological and related data centers took the opportunity to directly capture and
store these in their databases. Much was learned about automatic methods of collecting and
processing meteorological data in the late 1950s, a period that included the International
Geophysical Year and the establishment of the World Weather Watch. The WMO’s development
of international guidelines and standards for climate data management and data exchange
assisted NMHSs in organizing their data management activities and, less directly, also furthered
the development of regional and global databases. Today, the management of climate records
requires a systematic approach that encompasses paper records, microfilm/microfiche records
and digital records, where the latter include image files as well as the traditional alphanumeric
representation.

Before electronic computers, mechanical devices played an important part in the development
of data management. Calculations were made using comptometers, for example, with the results
being recorded on paper. A major advance occurred with the introduction of the Hollerith system
of punch cards, sorters and tabulators. These cards, with a series of punched holes recording
the values of the meteorological variables, were passed through the sorting and tabulating
machines enabling more efficient calculation of statistics.
The 1960s and 1970s saw several NMHSs implementing electronic computers and gradually the
information from many millions of punched cards was transferred to magnetic tape. These
computers were replaced with increasingly powerful mainframe systems and data were made
available online through developments in disk technology.

Aside from advances in database technologies, more efficient data capture was made possible
through the mid-to-late 1990s with an increase in automatic weather stations (AWSs), electronic
field books (i.e. on-station notebook computers used to enter, quality control and transmit
observations), the Internet and other advances in technology.
Not surprisingly, there are a number of trends already underway that suggest there are many
further benefits for NMHSs in managing data and servicing their clients. The Internet is already
delivering greatly improved data access capabilities and, providing security issues are managed,
we can expect major opportunities for data managers in the next five to ten years. In addition,
Open Source7 relational database systems may also remove the cost barriers to relational
databases for many NMHSs over this period.