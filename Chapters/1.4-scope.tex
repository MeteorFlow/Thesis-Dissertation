%trung tâm khí tượng thủy văn khu vực nam bộ

Dự án sẽ tập trung vào Thành phố Hồ Chí Minh, một đô thị lớn với môi trường khí hậu đặc biệt và có ảnh hưởng lớn đến cuộc sống hàng ngày của cộng đồng. 


Chúng tôi sẽ nghiên cứu và thu thập dữ liệu từ nhiều trạm ra-đa và trạm quan trắc khí tượng thuỷ văn trong thành phố để đảm bảo tính đa dạng và đại diện cho các điều kiện thời tiết địa phương. Sau khi đã thu thập đủ dữ liệu cần thiết, chúng tôi sẽ tiến hành thực hiện các bước tiền xử lý và chuẩn hoá dữ liệu để đảm bảo độ chính xác và tính nhất quán của tập dữ liệu. Cuối cùng, chúng tôi sẽ xây dựng cơ sở dữ liệu tích hợp để lưu trữ và quản lý thông tin từ các nguồn khác nhau, tạo ra một nguồn dữ liệu đồng nhất và đáng tin cậy. Nguồn thông tin tích hợp sẽ được sử dụng để cung cấp dữ liệu đa chiều và chi tiết về điều kiện thời tiết hiện tại và dự đoán ngắn hạn. Mục tiêu là giúp cộng đồng và các đơn vị liên quan chuẩn bị tốt hơn cho biến động thời tiết, đặc biệt là những thay đổi không dự đoán được.


Chúng tôi mong đợi rằng việc có một cơ sở dữ liệu tích hợp sẽ cải thiện hiệu quả của các mô hình dự báo thời tiết, giúp người dân và doanh nghiệp đối mặt với thời tiết khó khăn một cách thông minh và an toàn.

%\begin{itemize}
    %\item \textbf{Thu Thập Dữ Liệu Thực Tế:} \\
    %Tiến hành thu thập dữ liệu thực tế từ một số trạm ra-đa và quan trắc tại Tp. Hồ Chí Minh và [Thêm nội dung mở rộng nếu cần].
    
   % \item \textbf{Tiền Xử Lý và Chuẩn Hoá:} \\
    %Thực hiện các bước tiền xử lý và chuẩn hoá dữ liệu để đảm bảo độ chính xác và tính nhất quán.
    
    %\item \textbf{Xây Dựng Hệ Cơ Sở Dữ Liệu Tích Hợp:} \\
    %Tạo ra một hệ cơ sở dữ liệu tích hợp bao gồm cả dữ liệu số và hình ảnh để đáp ứng đầy đủ nhu cầu của các chuyên gia và nhà nghiên cứu.
%\end{itemize}
