\section{Problem statement}

In recognizing the need for National Meteorological Services (NMHSs) to improve their climate data and monitoring services, there is a deliberate focus on those aspects of climate data management wishing to make the transition to a modern climate database management system and, just as important, on what skills, systems and processes need to be in place to ensure that operations are sustained. In the context of the ever-growing complexity of climate change, the task of creating an integrated database for short-term forecasting systems in the fields of meteorology and hydrology poses a considerable challenge. The question at hand is how we can optimize the management of weather information from multiple sources and store it efficiently in a database. This optimization is crucial to ensure the provision of synchronized and high-quality information to support forecasting systems.

\section{Motivation}


Information about the weather has been recorded in manuscript form for many
centuries. The early records included notes on extreme and, sometimes,
catastrophic events and also on phenomena such as the freezing and thawing dates
of rivers, lakes and seas, which have taken on a higher profile with recent
concerns about climate change.

Specific journals for the collection and retention of climatological information
have been used over the last two or three centuries (WMO 2005). The development
of instrumentation to quantify meteorological phenomena and the dedication of
observers to maintaining methodical, reliable and well-documented records paved
the way for the organized management of climate data. Since the 1940s,
standardized forms and procedures gradually became more prevalent and, once
computer systems were being used by NMHSs, these forms greatly assisted the
computerized data entry process and consequently the development of computer
data archives. The latter part of the twentieth century saw the routine exchange
of weather data in digital form and many meteorological and related data centers
took the opportunity to directly capture and store these in their databases.
Much was learned about automatic methods of collecting and processing
meteorological data in the late 1950s, a period that included the International
Geophysical Year and the establishment of the World Weather Watch. The WMO’s
development of international guidelines and standards for climate data
management and data exchange assisted NMHSs in organizing their data management
activities and, less directly, also furthered the development of regional and
global databases. Today, the management of climate records requires a systematic
approach that encompasses paper records, microfilm/microfiche records and
digital records, where the latter include image files as well as the traditional
alphanumeric representation.

Before electronic computers, mechanical devices played an important part in the
development of data management. Calculations were made using comptometers, for
example, with the results being recorded on paper. A major advance occurred with
the introduction of the Hollerith system of punch cards, sorters and tabulators.
These cards, with a series of punched holes recording the values of the
meteorological variables, were passed through the sorting and tabulating
machines enabling more efficient calculation of statistics. The 1960s and 1970s
saw several NMHSs implementing electronic computers and gradually the
information from many millions of punched cards was transferred to magnetic
tape. These computers were replaced with increasingly powerful mainframe systems
and data were made available online through developments in disk technology.

Aside from advances in database technologies, more efficient data capture was
made possible through the mid-to-late 1990s with an increase in automatic
weather stations (AWSs), electronic field books (i.e. on-station notebook
computers used to enter, quality control and transmit observations), the
Internet and other advances in technology. Not surprisingly, there are a number
of trends already underway that suggest there are many further benefits for
NMHSs in managing data and servicing their clients. The Internet is already
delivering greatly improved data access capabilities and, providing security
issues are managed, we can expect major opportunities for data managers in the
next five to ten years. In addition, Open Source7 relational database systems
may also remove the cost barriers to relational databases for many NMHSs over
this period.

\section{Project Goal}

It is essential that both the development of climate databases and the
implementation of data management practices take into account the needs of the
existing, and to the extent that it is predictable, future data users. While at
first sight this may seem intuitive, it is not difficult to envisage situations
where, for example, data structures have been developed that omit data important
for a useful application or where a data centre commits too little of its
resources to checking the quality of data for which users demand high quality.

In all new developments, data managers should either attempt to have at least
one key data user as part of their project team or undertake some regular
consultative process with a group of user stakeholders. Data providers or data
users within the organization may also have consultative processes with end
users of climate data (or information) and data managers should endeavour to
keep abreast of both changes in needs and any issues that user communities have.
Put simply, data management requires awareness of the needs of the end users.

At present, the key demand factors for data managers are coming from climate
prediction, climate change, agriculture and other primary industries, health,
disaster/emergency management, energy, natural resource management (including
water), sustainability, urban planning and design, finance and insurance. Data
managers must remain cognizant that the existence of the data management
operation is contingent on the centre delivering social, economic and
environmental benefit to the user communities it serves. It is important,
therefore, for the data manager to encourage and, to the extent possible,
collaborate in projects which demonstrate the value of its data resource. Even
an awareness of studies that show, for example, the economic benefits from
climate predictions or the social benefits from having climate data used in a
health warning system, can be useful in reminding senior NMHS managers or
convincing funding agencies that data are worth investing in. Increasingly,
value is being delivered through integrating data with application models (e.g.
crop simulation models, economic models) and so integration issues should be
considered in the design of new data structures.

\section{Project Scope}

This project focuses on Ho Chi Minh City, a sprawling metropolis with a distinct
climate that significantly impacts the daily lives of its residents. To provide
the most valuable weather insights, we will implement a two-pronged approach.

Firstly, We will conduct research and collect data from the meteorological and
hydrological monitoring stations in the city to ensure diversity and
representation of local weather conditions. At the same time, we will proceed
with preprocessing and standardizing the data to ensure accuracy and consistency
of the dataset. Based on what we have collected, we will populate the data to
proper formats and structures.

Secondly, this project entails the development of a comprehensive data platform.
This platform will integrate a central database, designed to efficiently store
and manage weather information from diverse sources. By consolidating these
disparate data streams, we aim to create a unified and dependable resource for
weather data. This platform is anticipated to significantly enhance the workflow
of end-users and will be built with scalability in mind to accommodate future
growth in data volume and user demand.

\section{Stakeholders}

\subsection{The National Center for Hydrometeorological Forecasting (NCHMF)}

The National Center for Hydrometeorological Forecasting (NCHMF), abbreviated for
"Trung tâm Dự báo khí tượng thuỷ văn quốc gia" \ in Vietnamese, is an
organizational unit under the General Department of Meteorology and Hydrology,
Ministry of Natural Resources and Environment\cite{NMHS}. The National
Hydro-Meteorological Forecasting Center has several crucial missions, including
the establishment and presentation of standards and technical regulations for
meteorological and hydrological forecasting, the operation of the national
forecasting and warning system, monitoring and reporting on weather conditions
and climate change, issuing and disseminating forecast bulletins and warnings,
and participating in international meteorological agreements. Additionally, the
center is responsible for conducting research, application, and technology
transfer related to forecasting and warning, and implementing administrative
reform and anti-corruption measures. These key missions contribute significantly
to the center's role in ensuring public safety and providing timely essential
meteorological and hydrological information.

\subsection{Nha Be Weather Radar Station}

The Nha Be weather radar station is located on Nguyen Van Tao Street, Long Thoi
Commune, Nha Be District, Ho Chi Minh City. This radar station has been in
operation since 2004 and is managed by the Southern Region Hydro-Meteorological
Center. Its mission is to monitor and supervise weather phenomena within a
radius of 480 km from the center of Ho Chi Minh City, and to provide warnings
and forecasts for dangerous weather conditions such as storms, tropical
depressions, and thunderstorms within a radius of 300 km. Additionally, the
station serves the purpose of forecasting rain and heavy rain for the Ho Chi
Minh City area within a radius of approximately 120 km.

The Nha Be weather radar station is of the Doppler DWSR-2500C type, operating on
the C-band frequency. It is manufactured by the U.S. company EEC and has the
capability to scan signals within a radius of 300 km. In addition to being the
forecasting center for the entire Southern region and Ho Chi Minh City, the Nhà
Bè radar station also has the task of predicting rain and providing data for the
flood control center of Ho Chi Minh City.


\section{Solutions}

Our product, the Weather Data Platform (WDP), is designed to be a one-stop shop
for harnessing the power of weather data. It caters to the specific needs of a
wide range of users, including meteorologists, academics, researchers, and
developers from various fields. Whether you're a freelancer, a large
corporation, or a non-profit organization, WDP can be your central hub for
integrating, analyzing, and visualizing weather data – and much more.

\section{Achievements and Contribution}