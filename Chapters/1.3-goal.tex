%\quad Dự án tập trung vào việc nghiên cứu và xây dựng một cơ sở dữ liệu tích hợp, tổng hợp thông tin từ nhiều nguồn ở các trạm ra-đa và quan trắc khí tượng thuỷ văn, đặc biệt là tại Thành phố Hồ Chí Minh. Mục tiêu của chúng tôi là tạo ra một nguồn thông tin đáng tin cậy và toàn diện, hỗ trợ cho việc dự đoán hiệu quả các biến động thời tiết ngắn hạn.

It is essential that both the development of climate databases and the implementation of data
management practices take into account the needs of the existing, and to the extent that it is
predictable, future data users. While at first sight this may seem intuitive, it is not difficult to
envisage situations where, for example, data structures have been developed that omit data
important for a useful application or where a data centre commits too little of its resources to
checking the quality of data for which users demand high quality.

In all new developments, data managers should either attempt to have at least one key data
user as part of their project team or undertake some regular consultative process with a group of
user stakeholders. Data providers or data users within the organization may also have
consultative processes with end users of climate data (or information) and data managers
should endeavour to keep abreast of both changes in needs and any issues that user
communities have. Put simply, data management requires awareness of the needs of the end
users.

At present, the key demand factors for data managers are coming from climate prediction,
climate change, agriculture and other primary industries, health, disaster/emergency
management, energy, natural resource management (including water), sustainability, urban
planning and design, finance and insurance. Data managers must remain cognizant that the
existence of the data management operation is contingent on the centre delivering social,
economic and environmental benefit to the user communities it serves. It is important, 
therefore, for the data manager to encourage and, to the extent possible, collaborate in projects which
demonstrate the value of its data resource. Even an awareness of studies that show, for
example, the economic benefits from climate predictions or the social benefits from having
climate data used in a health warning system, can be useful in reminding senior NMHS
managers or convincing funding agencies that data are worth investing in. Increasingly, value is
being delivered through integrating data with application models (e.g. crop simulation models,
economic models) and so integration issues should be considered in the design of new data
structures.