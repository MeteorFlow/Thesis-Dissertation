
\begin{declaration}

    We guarantee that this research is our own, conducted under the supervision and guidance of
    Assoc. Prof. Dr. Lê Hồng Trang. The result of our research is legitimate and has not been published in any
    form prior to this. All materials used within this research are collected by ourselves, by
    various sources and are appropriately listed in the references section. In addition, within this
    research, we also used the results of several other authors and organizations. They have all been
    aptly referenced. In any case of plagiarism, we stand by our actions and are to be responsible for
    it. Ho Chi Minh City University of Technology therefore is not responsible for any copyright
    infringements conducted within our research

    \begin{flushright}

        \begin{tabular}{@{}c@{}}
            Ho Chi Minh City, \today \\
            \textbf{Team Members}    \\
            Trần Hà Tuấn Kiệt        \\
            Nguyễn Đức Thuỵ
        \end{tabular}

    \end{flushright}
\end{declaration}

\newpage

%-	Lời cảm ơn/ Lời ngỏ
\begin{acknowledgments}

    We, the research group consisting of two members, would like to send our thanks
    toward everyone who has contributed to the success of this thesis.


    First and foremost, we would like to express our gratitude to \Proc.
    The dedication and extensive knowledge of our supervisor not only guided us
    through the challenges of the project but also helped us develop research skills
    and critical thinking.

    We also want to extend special thanks to all the friends, colleagues,
    and family who supported us throughout the research process.
    The input and opinions of everyone enriched the content and quality of the project.

    Last, but certainly not least, we appreciate each other - research partners
    accompanying us in every step of the project.
    Our collaboration and joint contributions have resulted in a product
    that we take pride in.

    We believe that this project marks a significant step in our development
    and could not have been achieved without the support and contributions of everyone involved.


\end{acknowledgments}
\newpage
%-	Tóm tắt LV
\begin{abstr}
    In this document, we introduce a comprehensive proposal to integrate, coordinate,
    and monitor meteorological and hydrological data in Vietnam.
    The proposed system can serve as a foundation for building a national database specializing in meteorological information.

    To clarify our proposal, we have built a comprehensive pilot version,
    in which we have simulated the process of collecting and transmitting data
    from the weather station at Nha Be. At the same time, we focused on the process
    of converting and organizing data storage to ensure transparency and optimal performance
    in information processing.

    This research is a manifestation of significant efforts in optimizing the management of
    meteorological data in Vietnam. The proposed system not only addresses the challenges of
    integration and coordination but also lays the foundation for a robust national database,
    serving the diverse needs of research and applications in the field of meteorology.
\end{abstr}


