\begin{declaration}

Chúng tôi cam đoan rằng đồ án này là kết quả của công việc nghiên cứu của chúng tôi, được thực hiện dưới sự hướng dẫn tận tâm của \Proc tại \Uni. Chúng tôi xác nhận rằng tất cả các thành viên trong nhóm đã đóng góp công sức và kiến thức của mình để hoàn thành công trình này.

Chúng tôi cam đoan rằng đồ án này không phải là bản sao từ bất kỳ công trình nào khác và tất cả các nguồn thông tin đã được chú thích đầy đủ theo quy tắc trích dẫn khoa học. Mọi ý kiến, thông tin hay dữ liệu từ nguồn ngoại lai đều được công bố và thực hiện theo quy tắc đạo đức và quy định của \Uni.

Chúng tôi xác nhận rằng tất cả các thành viên trong nhóm đã thực hiện nghiên cứu này với lòng tận tụy và tính chân thành cao, tuân thủ mọi nguyên tắc đạo đức nghiên cứu khoa học. Chúng tôi cũng đảm bảo rằng không có hành vi gian lận hoặc vi phạm nào đã xảy ra trong quá trình thực hiện đồ án.

Chúng tôi chấp nhận mọi trách nhiệm và hậu quả pháp lý liên quan đến nội dung của đồ án và đồng thời biết ơn sự hướng dẫn, góp ý quý báu của giáo viên \Proc đã giúp chúng tôi hoàn thiện công trình này.

\begin{flushright}

    \begin{tabular}{@{}c@{}}
    Thành phố Hồ Chí Minh, \today \\
    \textbf{NGƯỜI THỰC HIỆN} \\
    Trần Hà Tuấn Kiệt \\
    Nguyễn Đức Thuỵ
    \end{tabular}

\end{flushright}
\end{declaration}

\newpage

%-	Lời cảm ơn/ Lời ngỏ
\begin{acknowledgments}

Chúng tôi, nhóm nghiên cứu gồm hai thành viên, xin gửi lời cảm ơn chân thành đến tất cả mọi người đã đóng góp vào thành công của đồ án này.

Đầu tiên và quan trọng nhất, chúng tôi muốn bày tỏ lòng biết ơn đến \Proc. Sự tận tâm và kiến thức sâu rộng của thầy không chỉ hướng dẫn chúng tôi qua những thách thức của đồ án mà còn giúp chúng tôi phát triển kỹ năng nghiên cứu và phê bình.

Chúng tôi cũng muốn bày tỏ lòng biết ơn đặc biệt đến tất cả những người bạn, đồng nghiệp, và gia đình đã hỗ trợ chúng tôi trong suốt quá trình nghiên cứu. Sự góp ý và ý kiến của mọi người đã làm giàu thêm nội dung và chất lượng của đồ án.

Cuối cùng, nhưng không kém phần quan trọng, chúng tôi cảm ơn nhau - đối tác nghiên cứu đồng hành trong mỗi bước của đồ án. Sự hợp tác và đóng góp chung của chúng tôi đã tạo nên một sản phẩm mà chúng tôi tự hào.

Chúng tôi tin rằng đồ án này là một bước tiến quan trọng trong sự phát triển của chúng tôi và không thể đạt được mà không có sự hỗ trợ và đóng góp của tất cả mọi người.

\end{acknowledgments}
\newpage
%-	Tóm tắt LV
\begin{abstr}
Trong tài liệu này, chúng tôi giới thiệu một đề xuất toàn diện để tích hợp, điều phối và giám sát dữ liệu khí tượng thủy văn tại Việt Nam. Hệ thống được đề xuất có thể làm nền tảng cho việc xây dựng một kho dữ liệu quốc gia chuyên sâu về thông tin khí tượng.

Để làm rõ đề xuất của chúng tôi, chúng tôi đã xây dựng một bản thử nghiệm toàn diện, trong đó chúng tôi đã thực hiện mô phỏng về quy trình thu thập và truyền phát dữ liệu từ trạm thời tiết tại Nhà Bè. Đồng thời, chúng tôi đã tập trung vào quá trình chuyển đổi và tổ chức lưu trữ dữ liệu để đảm bảo sự minh bạch và hiệu suất tối ưu trong quá trình xử lý thông tin.

Bài nghiên cứu này là biểu hiện của những nỗ lực đáng kể trong việc tối ưu hóa quản lý dữ liệu khí tượng tại Việt Nam. Hệ thống đề xuất này không chỉ giải quyết các thách thức về tích hợp và điều phối mà còn đặt nền tảng cho một kho dữ liệu quốc gia vững mạnh, phục vụ cho những nhu cầu đa dạng của nghiên cứu và ứng dụng trong lĩnh vực khí tượng.

\end{abstr}	


