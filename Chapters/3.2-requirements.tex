\subsection{Functional Requirement}

In the initial phase of the data processing workflow, \textbf{data ingestion}
emerges as a pivotal undertaking, wherein data is systematically amassed from a
spectrum of heterogeneous sources, spanning from \textbf{Application Programming
    Interfaces (APIs)} to repositories housing \textbf{government data}. This
foundational process assumes paramount significance as it sets the groundwork
for the acquisition of an extensive array of datasets essential for
comprehensive analytical endeavors.

Following data ingestion, the subsequent stage revolves around \textbf{data
    storage} and \textbf{management}. This facet encompasses the orchestration of
repositories, potentially ranging from conventional databases to more
contemporary constructs like \textbf{data lakes}. Such infrastructural
modalities afford robust frameworks that adeptly cater to the exigencies imposed
by the management of copious volumes of data, thereby ensuring efficiency and
efficacy in handling data resources.

Concomitant with data management, \textbf{data analysis} and
\textbf{visualization} procedures are meticulously tailored to cater to the
discerning requirements of stakeholders entrenched within the meteorological
domain and academic research milieu. These analytical tools serve as
indispensable instruments, facilitating granular scrutiny and cogent
interpretation of meteorological datasets, thereby fostering informed
decision-making and scholarly inquiry.

Facilitating seamless interoperability with external platforms and augmenting
the system's functionality through synergistic engagement with \textbf{APIs}
represent pivotal components of the overarching operational paradigm. Such
integrative endeavors engender an ecosystem conducive to harmonious interactions
with an array of technological adjuncts, thereby amplifying the system's utility
and efficacy.

Furthermore, the system incorporates robust \textbf{user management} protocols
and \textbf{access control mechanisms} to ensure the sanctity of data assets,
conferring access exclusively to duly authorized personnel. By meticulously
safeguarding data integrity and confidentiality, these provisions engender an
environment predicated upon trust and accountability, thereby fortifying the
organizational edifice.

Finally, the culminating facets of \textbf{reporting} and \textbf{alerting
    functionalities} emerge as indispensable conduits for the expeditious
dissemination of critical insights and information to an array of stakeholders,
encompassing end-users, organizational hierarchies, and strategic
decision-makers. These functionalities serve as conduits for the expeditious
transmission of salient information, thereby empowering stakeholders to navigate
dynamic operational landscapes with acumen and alacrity.

\subsection{Non-Functional Requirements}

In the realm of technological platforms, the facet of \textbf{Reliability}
stands as a cornerstone, necessitating unwavering commitment towards ensuring
persistent high availability and steadfast operational stability. Paramount to
this pursuit is the unyielding dedication towards upholding \textbf{data
    integrity}, \textbf{security}, and \textbf{privacy} as cardinal imperatives,
safeguarding both user information and system data against potential
vulnerabilities and breaches.

Moreover, the imperative of \textbf{Scalability} looms large, compelling the
platform to confront the exigencies posed by burgeoning volumes of data. In this
regard, an adept embrace of \textbf{cloud-native technologies} and the paradigm
of \textbf{containerization}, exemplified by stalwarts such as Docker and
Kubernetes, emerges as indispensable. Such technological modalities proffer the
requisite infrastructure agility, facilitating seamless and expedient scaling of
both infrastructure and applications alike, thereby ensuring commensurate
responsiveness to evolving operational demands.

Further accentuating the discourse, the criterion of \textbf{Performance}
assumes primacy, necessitating the provision of expeditious data processing and
analysis capabilities. This imperative is underscored by the imperatives of
real-time decision-making imperatives and the imperative of managing large-scale
data operations with finesse and alacrity.

In tandem, the imperative of \textbf{Usability} assumes paramount significance,
underscoring the imperative of furnishing an interface that is as intuitive as
it is user-friendly. A bespoke interface, meticulously tailored to cater to the
diverse exigencies of disparate user cohorts, stands as a testament to the
platform's commitment towards ensuring unfettered ease of access and usability.

Moreover, the aspect of \textbf{Security} mandates robust fortifications,
encompassing both data protection protocols and stringent access control
mechanisms. This multifaceted approach is calibrated to thwart the machinations
of unauthorized access and avert the specter of potential data breaches, thereby
undergirding the sanctity of the platform's operational integrity.

Simultaneously, adherence to \textbf{Compliance} imperatives emerges as
non-negotiable, enjoining scrupulous conformity with a panoply of regulatory
stipulations, data privacy statutes, and overarching compliance requisites. Such
fidelity towards regulatory benchmarks serves as an indispensable bulwark,
ensuring that the platform operates within the hallowed precincts of legal and
ethical propriety.

Lastly, expounding upon the contours of \textbf{scaling technology and
    infrastructure}, the platform espouses an ethos predicated upon the tenets of
\textbf{cloud-native architectures} and containerization technologies. This
strategic pivot affords the platform the requisite nimbleness, facilitated by
the judicious employment of auto-scaling capabilities and the elastic resources
proffered by cloud platforms. Furthermore, the adoption of a
\textbf{microservices architecture} and the decoupling of components augur well
for bolstering the platform's resilience and flexibility, enabling independent
scaling of discrete services or modules, and thus engendering a paradigm of
operational dynamism and adaptability.

\subsection{Data Requirements}
In the intricate domain of radar data management, the foundational phase
commences with an exhaustive \textbf{cleaning process}. This pivotal endeavor is
indispensable, serving as the vanguard against the encroachment of
\textbf{noise} and extraneous data, thereby ensuring the veracity and
reliability of the data corpus. The imperative of cleansing radar files assumes
paramount significance, poised as it is to ready them for the rigors of more
intricate processing and analysis, which constitute linchpins in facilitating
accurate meteorological appraisals.

Subsequent to the meticulous cleansing regimen, the radar data traverses a
sophisticated trajectory, wherein it becomes ensconced within a nuanced
\textbf{processing routine}. At this juncture, specialized algorithms assume the
mantle, orchestrating a symphony of analyses to distill meaningful patterns and
insights from the data trove. This transformative phase is pivotal, transmuting
raw data into actionable intelligence, thereby furnishing decision-makers with a
bedrock of dependable insights.

Post the crucible of processing, the data finds sanctuary within the confines of
\textbf{blobs} nestled within an object storage apparatus, typically epitomized
by \textbf{Minio}. This methodical selection of storage modality is predicated
upon its inherent scalability and facile accessibility, attributes that are
deemed indispensable in wrangling the prodigious volumes of data bequeathed by
radar systems. Moreover, the architecture of blob storage engenders a milieu
conducive to expeditious data retrieval and management, adeptly accommodating
the dynamic access patterns germane to meteorological exigencies.

In a bid to further fortify the edifice of data handling efficacy, the apparatus
embraces the instantiation of a \textbf{caching mechanism}. This strategic
deployment bestows upon the system the capability to temporarily harbor recently
accessed data, thereby effectuating a palpable reduction in retrieval latencies.
This feature assumes outsized significance, particularly in scenarios of
heightened exigency, such as severe weather events, wherein the expeditious
access to data assumes the mantle of existential import.

The hallowed precincts of processed and stored radar data resonate with a
multifaceted utility, traversing a panoply of applications. Meteorologists,
ensconced within the annals of weather forecasting and climate modeling, find
themselves inexorably tethered to this fount of data-driven insights.
Concurrently, the pantheon of emergency response entities stands as ardent
beneficiaries, drawing succor from the real-time data streams that undergird
effective decision-making in the crucible of weather-related emergencies.
Furthermore, the expansive vista of scientific inquiry finds solace in this rich
tapestry of data, wielding it as a potent instrument in unraveling the enigmatic
contours of climate patterns and atmospheric dynamics.
