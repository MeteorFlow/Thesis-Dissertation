\subsection{Functional Requirement}

\textbf{Data ingestion} is the initial step where data is gathered from various
sources, including \textbf{APIs} and \textbf{government data repositories}. This
process is critical for accumulating the diverse datasets required for thorough
analysis.

The next phase involves \textbf{data storage} and \textbf{management}. This may
be carried out using databases or \textbf{data lakes}, providing robust
solutions for handling extensive volumes of data effectively.

\textbf{Data analysis} and \textbf{visualization} are tailored specifically to
support the needs of \textbf{meteorologists} and \textbf{researchers}. These
tools are crucial for the detailed examination and interpretation of
meteorological data.

\textbf{Integration} with third-party platforms and \textbf{APIs} enhances the
system's functionality, enabling seamless interactions with other technological
tools.

The system incorporates \textbf{user management} and \textbf{access control
mechanisms} to ensure data security and accessibility only to authorized
personnel, maintaining data integrity and confidentiality.

Lastly, the system includes \textbf{reporting} and \textbf{alerting
functionalities} that are vital for communicating essential information to
users, stakeholders, and decision-makers in a timely and efficient manner.

\subsection{Non-Functional Requirements}

\textbf{Reliability} is a critical component, requiring the platform to ensure
high availability and stable operation consistently. It is also crucial to
maintain \textbf{data integrity}, \textbf{security}, and \textbf{privacy} as top
priorities to protect user information and system data.

\textbf{Scalability} needs to be addressed to handle increasing volumes of data.
The system should have the ability to scale efficiently, leveraging
\textbf{cloud-native technologies} and \textbf{containerization} such as Docker
and Kubernetes, to facilitate easy scaling of infrastructure and applications.

In terms of \textbf{performance}, the platform must offer rapid data processing
and analysis capabilities to meet the demands of real-time decision making and
large-scale data handling.

\textbf{Usability} is essential, with a focus on providing an intuitive and
\textbf{user-friendly interface} that is tailored for different user groups to
ensure ease of use and accessibility.

\textbf{Security} measures must include robust data protection and access
control mechanisms to safeguard against unauthorized access and potential data
breaches.

\textbf{Compliance} with applicable regulatory standards, data privacy laws, and
compliance requirements is mandatory to ensure that the platform operates within
legal constraints.

Lastly, expanding on \textbf{scaling technology and infrastructure}, the
platform should adopt \textbf{cloud-native architectures} and containerization
technologies to enable easy scaling. Utilizing auto-scaling capabilities and
elastic resources provided by cloud platforms allows for dynamic adjustment of
capacity based on demand. Additionally, implementing a \textbf{microservices
architecture} and decoupled components will enable independent scaling of
individual services or modules, enhancing overall system resilience and
flexibility.

\subsection{Data Requirements}
% TODO: How data is stored, processed and 

% How radar file are processed, cleaned, stored and used Storing data as blob -
% Radar Object (Minio) Caching
The initial step in handling radar data involves a comprehensive
\textbf{cleaning process}. This is crucial to remove any \textbf{noise} and
extraneous data, ensuring the accuracy and reliability of the data. Cleaning
radar files is essential as it preps them for more complex processing and
analysis, which are critical for accurate meteorological assessments.

Once the radar data has been cleaned, it undergoes a sophisticated
\textbf{processing routine}. Specialized algorithms analyze the data to extract
meaningful patterns and information, structuring it into a format that can be
easily utilized in further analyses. This step is vital for transforming raw
data into actionable insights that can be relied upon for decision-making.

After processing, the data is stored as \textbf{blobs} in an object storage
system, typically \textbf{Minio}. This method of storage is selected for its
scalability and ease of access, which are necessary attributes for managing the
voluminous data generated by radar systems. Blob storage also facilitates the
efficient retrieval and management of data, supporting dynamic access patterns
required by meteorological applications.

To further enhance data handling efficiency, a \textbf{caching mechanism} is
implemented. This system stores recently accessed data temporarily,
significantly reducing retrieval times and ensuring that frequently needed
information is readily available. This feature is particularly important during
critical situations, such as severe weather events, where timely data access is
crucial.

The processed and stored radar data is extensively used across various
applications. Meteorologists rely on this data for accurate \textbf{weather
forecasting} and \textbf{climate modeling}. It is also crucial for
\textbf{emergency response teams} who depend on real-time data for effective
decision-making during weather-related emergencies. Furthermore, the data
supports a broad range of scientific research, aiding in the study of climate
patterns and atmospheric conditions.
