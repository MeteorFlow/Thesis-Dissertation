% TODO: How data are stored? Storage - structured, unstructured, semi-structured
% Naming convention Data format

\subsection{Data modeling}

\subsection{Blob storage}
% Naming convention Data format

Blob storage is an optimal solution for storing large, unstructured data files
such as radar object files. This type of storage system is ideal for handling
massive amounts of data generated by radar systems, allowing for scalable and
cost-effective data management.

\textbf{Naming Convention:}
When storing radar data in blob storage, establishing a systematic naming
convention is essential for organization and retrieval efficiency. A recommended
approach includes starting with the date of data collection in \texttt{YYYYMMDD}
format to sort files chronologically. This is followed by the time the data was
recorded, noted in \texttt{HHMMSS} format, which is crucial for distinguishing
between multiple scans taken on the same day. Additionally, including a short
identifier or code for the radar station or specific radar equipment used is
important, especially when data is collected from multiple sources. Specifying
the type of radar data, such as reflectivity, velocity, or dual-polarization,
helps in filtering and processing specific data types for analysis. Finally, if
applicable, including a version number ensures that users access the most recent
or appropriate version of the data. An example naming convention could be
\texttt{20240520\_153000\_KATX\_reflectivity\_v1.nc}.

\textbf{Data Format:}
Choosing the right data format is critical for the usability and accessibility
of radar data. The formats should support the complexity of the data and
facilitate efficient processing and analysis. Popular formats include NetCDF,
which stands for Network Common Data Form and is widely used in meteorology for
its ability to handle large datasets efficiently. HDF5, or Hierarchical Data
Format version 5, is another excellent choice for handling large, complex
datasets, offering a more complex internal structure for detailed metadata
storage and fine control over data accessibility. CF/Radial, specifically
designed for radar data, is a convention of NetCDF tailored for storing radial
data from weather radars and supports the storage of metadata such as radar
calibration information, scan strategies, and geolocation data.

\textbf{Security and Compliance:}
Ensure that the storage solution complies with data security regulations and
best practices. Implement access controls and data encryption to protect
sensitive information from unauthorized access.

\subsection{Caching}

