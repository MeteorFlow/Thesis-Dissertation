%Xây dựng cơ sở dữ liệu tích hợp cho các hệ thống dự báo ngắn hạn trong khí tượng thuỷ văn là một phần quan trọng trong việc cải thiện khả năng dự báo thời tiết và dự báo thủy văn. Để đáp ứng sự phức tạp ngày càng tăng của mô hình dự báo và cung cấp thông tin chính xác cho người dùng cuối, việc tích hợp dữ liệu từ nhiều nguồn và lưu trữ chúng trong cơ sở dữ liệu hiệu quả là không thể thiếu.

% Trong bối cảnh biến đổi khí hậu ngày càng diễn biến phức tạp, việc xây dựng cơ sở dữ liệu tích hợp cho các hệ thống dự báo ngắn hạn trong lĩnh vực khí tượng thuỷ văn trở thành một thách thức quan trọng. Bài toán đặt ra là làm thế nào chúng ta có thể tối ưu hóa quản lý thông tin thời tiết, từ nhiều nguồn và lưu trữ chúng trong một cơ sở dữ liệu hiệu quả, nhằm cung cấp nguồn thông tin đồng bộ và chất lượng cao để hỗ trợ các hệ thống dự báo.

The National Center for Hydrometeorological Forecasting (NCHMF), abbreviated for "Trung tâm Dự báo khí tượng thuỷ văn quốc gia" in Vietnamese, is an organizational unit under the General Department of Meteorology and Hydrology, Ministry of Natural Resources and Environment\cite{NMHS}. The National Hydro-Meteorological Forecasting Center has several crucial missions, including the establishment and presentation of standards and technical regulations for meteorological and hydrological forecasting, the operation of the national forecasting and warning system, monitoring and reporting on weather conditions and climate change, issuing and disseminating forecast bulletins and warnings, and participating in international meteorological agreements. Additionally, the center is responsible for conducting research, application, and technology transfer related to forecasting and warning, and implementing administrative reform and anti-corruption measures. These key missions contribute significantly to the center's role in ensuring public safety and providing essential meteorological and hydrological information.

There is a deliberate focus on those aspects of climate data management that are of interest to
NMHSs wishing to make the transition to a modern climate database management system and,
just as important, on what skills, systems and processes need to be in place to ensure that
operations are sustained. In the context of the ever-growing complexity of climate change, the task of creating an integrated database for short-term forecasting systems in the fields of meteorology and hydrology poses a considerable challenge. The question at hand is how we can optimize the management of weather information from multiple sources and store it efficiently in a database. This optimization is crucial to ensure the provision of synchronized and high-quality information to support forecasting systems.
